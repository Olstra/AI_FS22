\documentclass[11pt]{scrartcl}

\usepackage[utf8]{inputenc}
\usepackage[T1]{fontenc}
\usepackage[english]{babel}
\usepackage{lmodern}
\usepackage{graphicx}
\usepackage{listings}
\usepackage{xspace}
\usepackage[a4paper,lmargin={2cm},rmargin={2cm},tmargin={2.5cm},bmargin = {2.5cm},headheight = {4cm}]{geometry}
\usepackage{amsmath,amssymb,amstext,amsthm}
\usepackage[shortlabels]{enumitem}
\usepackage[headsepline]{scrlayer-scrpage} 
\pagestyle{scrheadings} 
\usepackage{titling}
\usepackage{etoolbox}
\usepackage{tikz}
\usetikzlibrary{shapes, arrows, calc, automata, arrows.meta, positioning,decorations.pathmorphing,backgrounds,decorations.markings,decorations.pathreplacing, graphs}

\tikzset{% 
    initial text={},
    state/.style={circle, draw, minimum size=.6cm},
    every initial by arrow/.style={-stealth},
    every loop/.append style={-stealth},
    >=stealth
}

\ohead{\theauthor}
\ihead{Introduction to Artificial Intelligence, Spring 2022, Sheet \thesheetnr}

% Sheet number
\newcounter{sheetnr}
\newcommand{\sheetnr}[1]{\setcounter{sheetnr}{#1}}

% Exercise environments
\newenvironment{exercise}[2][]{\section*{Exercise \thesheetnr.#2\expandafter\ifstrempty\expandafter{#1}{}{\ (#1)}}}{}
\newenvironment{subexercises}{\begin{enumerate}[a), font=\bfseries, wide, labelindent=0pt]}{\end{enumerate}}
\newenvironment{subsubexercises}{\begin{enumerate}[i), font=\bfseries, wide, labelindent=0pt]}{\end{enumerate}}



%% Examples
\newcommand{\Reach}{\problem{Reach}}


% Anpassen --> %
\author{Oliver Strassmann \\
        Julia Kostadinova \\
        Alessio Brazerol}
\sheetnr{4}
% <-- Anpassen %

\begin{document}

	\begin{exercise}[Formalization]{1}
	    \begin{center}
            $\mathcal{C} = \langle V, dom, C \rangle$
        \end{center}
        \begin{itemize}
            \item $V = \{v_{ij} | i \in \{A,B,C\} \land j \in \{1,2,3\}\}$
            \item $dom(v) = \{1,..,9\} \forall v \in V$
            \item unary constraints: 
                \begin{itemize}
                    \item $c_{v_{A1}} = \{1\}$
                    \item $c_{v_{A3}} = \{3\}$
                    \item $c_{v_{B2}} = \{2\}$
                \end{itemize}
            \item binary constraints:
                \begin{itemize}
                    \item $c_{v_{ij}v_{i'j'}} = \langle(v_{ij},v_{i'j'}), \{(a,b) \in \{1,...,3\}^2 | a \neq b\}\rangle$ for all $v_{ij},v_{i'j'} \in V$ with $i = i'$, xor $j = j'$
                \end{itemize}
        \end{itemize}

    \end{exercise}
    
    \begin{exercise}[Consistencies]{2}
        \begin{subexercises}
            \item[(a)] yes, because there are no values of the domains of $\mathbf{C}$ that are in conflict with a unary constraint on v.
            \begin{itemize}
                \item $dom(x) = \{2,4,5\} \subset \{2x' | x' \in \mathbb{N}\}$
                \item $dom(y) = \{1,4,9\} \subset \{y \in \mathbb{N} | y \neq 3\}$
                \item $dom(z) = \{0,1,2,3\} \subset \{z \in \mathbb{N} | z < 4\}$
            \end{itemize}
            \item[(b)] No, because there exists no $y\in dom(y)$ s.t. $y = 0^2$. $0\in dom(z)$ and for $c_5'$ this has to hold.
            \item[(c)] Yes, because for every consistent assignment $(a,b) \in \{(dom(x),dom(y)),(dom(y),dom(z)), (dom(z),dom(x))\}$ we find a consistent assignment $c$ in the not used domain. 
            \item[(d)] No, because $\mathbf{C}$ is not 2-consistent and thus not strongly 3-consistent.
            
        \end{subexercises}
    \end{exercise}
    
    \begin{exercise}[Consistencies]{3}
        \begin{subexercises}
            \item[(a)] Assignment order left to right \\
            \begin{tikzpicture}
            
                \tikzset{
                    node distance = 1.5cm,
                }
                
                \node [state, rectangle, fill=white] (s) { };
                
                %% v_1
                \node[state, color=white, fill=black, below of=s] (11) {$b$};
                \node[state, color=white, fill=black, right of=11] (12) {$g$};
                
                \path[] (s) edge[] node{} (11);
                \path[] (s) edge[] node{} (12);
                
                %% v_2
                \node[state, color=white, fill=black, below of=11] (21) {$g$};
                \node[state, rectangle, color=white, fill=red, left of=21] (22) {$b$};
                \node[state, color=white, fill=black, right of=21] (23) {$b$};
                
                \path[] (11) edge[] node{} (21);
                \path[] (11) edge[] node{} (22);
                \path[] (12) edge[] node{} (23);
                
                %% v_3
                \node[state, color=white, fill=black, below of=21] (31) {$r$};
                \node[state, rectangle, color=white, fill=red, left of=31] (32) {$b$};
                \node[state, color=white, fill=black, right of=31] (33) {$b$};
                
                \path[] (21) edge[] node{} (31);
                \path[] (21) edge[] node{} (32);
                \path[] (23) edge[] node{} (33);
                
                 %% v_4
                \node[state, color=white, fill=black, below of=31] (41) {$r$};
                \node[state, rectangle, color=white, fill=red, left of=41] (42) {$b$};
                \node[state, color=white, fill=black, right of=41] (43) {$b$};
                
                \path[] (31) edge[] node{} (41);
                \path[] (31) edge[] node{} (42);
                \path[] (33) edge[] node{} (43);
                
                %% v_5
                \node[state, color=white, fill=black, below of=41] (51) {$g$};
                \node[state, color=white, fill=black, left of=51] (52) {$b$};
                \node[state, rectangle, color=white, fill=red, below of=43] (53) {$b$};
                \node[state, color=white, fill=black, right of=53] (54) {$g$};
                
                \path[] (41) edge[] node{} (51);
                \path[] (41) edge[] node{} (52);
                \path[] (43) edge[] node{} (53);
                \path[] (43) edge[] node{} (54);
                
                %% v_6
                \node[state, color=white, fill=black, below of=51] (61) {$r$};
                \node[state, rectangle, color=white, fill=red, left of=61] (62) {$g$};
                \node[state, color=white, fill=black, left of=62] (63) {$y$};
                \node[state, color=white, fill=black, left of=63] (64) {$r$};
                \node[state, rectangle, color=white, fill=red, left of=64] (65) {$g$};
                \node[state, color=white, fill=black, right of=61] (66) {$y$};
                
                \node[state, rectangle, color=white, fill=red, below of=54] (67) {$g$};
                \node[state, color=white, fill=black, right of=67] (68) {$r$};
                
                \path[] (51) edge[] node{} (61);
                \path[] (51) edge[] node{} (62);
                \path[] (51) edge[] node{} (66);
                \path[] (52) edge[] node{} (63);
                \path[] (52) edge[] node{} (64);
                \path[] (52) edge[] node{} (65);
                \path[] (54) edge[] node{} (67);
                \path[] (54) edge[] node{} (68);
                
                %% v_7
                
                \node[state, rectangle, color=white, fill=red, below of=66] (71) {$b$};
                \node[state, rectangle, color=white, fill=red, right of=71] (72) {$r$};
                
                \node[state, rectangle, color=white, fill=red, below of=61] (73) {$r$};
                \node[state, rectangle, color=white, fill=red, left of=73] (74) {$b$};
                
                \node[state, rectangle, color=white, fill=red, below of=63] (75) {$r$};
                \node[state, rectangle, color=white, fill=red, left of=75] (76) {$b$};
                
                 \node[state, rectangle, color=white, fill=red, below of=65] (77) {$r$};
                \node[state, rectangle, color=white, fill=red, left of=77] (78) {$b$};
                
                \node[state, rectangle, color=white, fill=red, below of=68] (79) {$b$};
                \node[state, color=white, fill=green, right of=79] (80) {$r$};
                
                \path[] (66) edge[] node{} (71);
                \path[] (66) edge[] node{} (72);
                
                \path[] (61) edge[] node{} (73);
                \path[] (61) edge[] node{} (74);
                
                \path[] (63) edge[] node{} (75);
                \path[] (63) edge[] node{} (76);
                
                \path[] (64) edge[] node{} (77);
                \path[] (64) edge[] node{} (78);
                
                \path[] (68) edge[] node{} (79);
                \path[] (68) edge[] node{} (80);
                
                \node[state, color=white, text=black, fill=white, left of=78] (81) {$v_7$};
                \node[state, color=white, text=black, fill=white, above of=81] (82) {$v_6$};
                \node[state,color=white, text=black, fill=white, above of=82] (83) {$v_5$};
                \node[state, color=white, text=black, fill=white, above of=83] (84) {$v_4$};
                \node[state, color=white, text=black, fill=white, above of=84] (85) {$v_3$};
                \node[state, color=white, text=black, fill=white, above of=85] (86) {$v_2$};
                \node[state, color=white, text=black, fill=white, above of=86] (87) {$v_1$};
                
                \path[] (81) edge[] node{} (82);
                \path[] (82) edge[] node{} (83);
                \path[] (83) edge[] node{} (84);
                \path[] (84) edge[] node{} (85);
                \path[] (85) edge[] node{} (86);
                \path[] (86) edge[] node{} (87);
                
                
                
            \end{tikzpicture}
            
            
            \item[(b)] variable ordering from top to bottom and value ordering from left to right.
            \begin{center}
                \begin{tabular}{ c c }
                    variable ordering & value  ordering \\
                    \hline\hline
                    $v_7$ & r < b \\ 
                    
                    $v_2$ & b < g \\ 
                    
                    $v_5$ & g < b \\
                    
                    $v_3$ & b < r \\
                    
                    $v_4$ & b < r \\
                    
                    $v_1$ & g < b < r \\
                    
                    $v_6$ & r < g < b
                \end{tabular}
            \end{center} 
            We choose $v_5$ before $v_4$ because this will allow the value ordering for $v_4$ be the optimal one. We can do this because $v_4$ and $v_5$ are tied for the third place using the minimal remaining value heuristic.
            We choose the best value ordering for $v_1, v_3, v_4, v_5, v_6$ because the least constraining value heuristic results in a tie for these values. \\
            Yes, the resulting search tree would be faster, because the first value we choose for variable $v_i$ is always the same as for the solution in Ex. 4.3 (a).
            \item[(c)] If we have a partial assignment and choose variables based on most constrained, we will reach a conflicting variable in fewer steps if the assignment is conflicting. This is because the less constrained variables will most of the time have a possible assignment, so if we assign values to them first we have to go through them until we reach the more constrained variables that will produce a conflict. \\
            It makes sense to pick the least constraining value because this means we are more likely to find a possible assignment for the other variables and thus more likely to find a solution.
            
        \end{subexercises}
    \end{exercise}
    
    \begin{exercise}[Consistencies]{4}
        \begin{subexercises}
            \item[(a)] $ $\\
            \begin{tikzpicture}
            
                \tikzset{
                    node distance = 3.75cm,
                }
                
                \node [state, rectangle, fill=white] (s) { };
                
                %% v_2
                \node[state, rectangle, color=white, fill=red, below of=s, align=center] (21) {$1$ \\ $dom(v_1) = \emptyset$ \\ $dom(v_4) = \{1\}$};
                \node[state, color=white, fill=black, right of=21, align=center] (22) {$2$ \\ $dom(v_1) = \{1\}$ \\ $dom(v_3) = \{2\}$ \\ $dom(v_4) = \{3, 4\}$};
                
                \path[] (s) edge[] node{} (21);
                \path[] (s) edge[] node{} (22);
                
                %% v_1
                \node[state, rectangle, color=white, fill=red, below of=22, align=center] (11) {$1$ \\ $dom(v_3) = \emptyset$};
                
                \path[] (22) edge[] node{} (11);
            \end{tikzpicture}
            \item[(b)] $ $\\
            \begin{itemize}
                \item[Precondition:] 
                \begin{itemize}
                    \item[] $x_i = null$
                    \item[] $x_j = null$
                    \item[] $dom(x_1) = \{1,2\}$
                    \item[] $dom(x_2) = \{1,2\}$
                    \item[] $dom(x_3) = \{2,3,4\}$
                    \item[] $dom(x_4) = \{1,2,3\}$
                \end{itemize}
                
                \item[iteration 1:] 
                \begin{itemize}
                    \item[] $x_i = 1$
                    \item[] $x_j = 2$
                    \item[] $dom(x_1) := \{1\}$
                \end{itemize}
                
                \item[iteration 2:] 
                \begin{itemize}
                    \item[] $x_i = 2$
                    \item[] $x_j = 1$
                    \item[] $dom(x_2) := \{2\}$
                \end{itemize}
                
                \item[iteration 3:] 
                \begin{itemize}
                    \item[] $x_i = 1$
                    \item[] $x_j = 3$
                    \item[] $dom(x_1) := \emptyset$
                \end{itemize}
                \item[] The network is inconsistent!
                
            \end{itemize}
        \end{subexercises}
    \end{exercise}




\end{document}
