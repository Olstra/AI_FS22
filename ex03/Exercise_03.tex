\documentclass[11pt]{scrartcl}

\usepackage[utf8]{inputenc}
\usepackage[T1]{fontenc}
\usepackage[english]{babel}
\usepackage{lmodern}
\usepackage{graphicx}
\usepackage{listings}
\usepackage{xspace}
\usepackage[a4paper,lmargin={2cm},rmargin={2cm},tmargin={2.5cm},bmargin = {2.5cm},headheight = {4cm}]{geometry}
\usepackage{amsmath,amssymb,amstext,amsthm}
\usepackage[shortlabels]{enumitem}
\usepackage[headsepline]{scrlayer-scrpage} 
\pagestyle{scrheadings} 
\usepackage{titling}
\usepackage{etoolbox}
\usepackage{tikz}
\usetikzlibrary{shapes, arrows, calc, automata, arrows.meta, positioning,decorations.pathmorphing,backgrounds,decorations.markings,decorations.pathreplacing, graphs}

\tikzset{% 
    initial text={},
    state/.style={circle, draw, minimum size=.6cm},
    every initial by arrow/.style={-stealth},
    every loop/.append style={-stealth},
    >=stealth
}

\ohead{\theauthor}
\ihead{Introduction to Artificial Intelligence, Spring 2022, Sheet \thesheetnr}

% Sheet number
\newcounter{sheetnr}
\newcommand{\sheetnr}[1]{\setcounter{sheetnr}{#1}}

% Exercise environments
\newenvironment{exercise}[2][]{\section*{Exercise \thesheetnr.#2\expandafter\ifstrempty\expandafter{#1}{}{\ (#1)}}}{}
\newenvironment{subexercises}{\begin{enumerate}[a), font=\bfseries, wide, labelindent=0pt]}{\end{enumerate}}
\newenvironment{subsubexercises}{\begin{enumerate}[i), font=\bfseries, wide, labelindent=0pt]}{\end{enumerate}}



%% Examples
\newcommand{\Reach}{\problem{Reach}}


% Anpassen --> %
\author{Oliver Strassmann \\
        Julia Kostadinova \\
        Alessio Brazerol}
\sheetnr{3}
% <-- Anpassen %

\begin{document}

	\begin{exercise}[Heuristics]{1}
        \begin{subexercises}
            \item The wolf, goat, cabbage problem can be     described as:
            The farmer can move with or without an object (wolf, goat, cabbage) from one side of the river the another side if
            \begin{itemize}
                \item[1.] the farmer and the object are on the same side of the river,
                \item[2.] the wolf and the goat are not both on the opposite side of the farmer after moving
                \item[3.] the goat and the cabbage are not both on the opposite side of the farmer after moving
            \end{itemize}
                We can generate a new related problem:
            \begin{itemize}
                \item The farmer can move the objects or nothing from one side of the river to another side if the farmer and the object are on the same side of the river.
            \end{itemize}
            If we once again describe a state as a vector $s = (s_1, s_2, s_3, s_4) \in \{0,1\}^4$. And also only look at valid states in the original problem (states that do not invalidate any restrictions). We say, that $s_1$ describes the position of the farmer, $s_2$ the position of the wolf, $s_3$ the position of the goat, and $s_4$ the position of the cabbage. A $"1"$ indicates that the farmer or object is on the starting shore, a $"0"$ indicates that it is on the goal shore. So for example, the state $s=(0,1,0,1)$ describes the state where the farmer and the goat are on the goal shore and the wolf and the cabbage still on the starting shore. \\
            
            We can then define a heuristic $h$ that describes the minimal number of passages needed for the farmer to bring every object to the goal shore. \\
            
            \begin{center}
                \item $h(n) = 
                    \begin{cases}
                        0,& \text{if } n = (0,0,0,0)\\
                        1,& \text{if } n = (1,0,0,0)\\ 
                        (1 - n_1) + 2(n_2 + n_3 + n_4) - 1,& \text{otherwise}
                    \end{cases}$ \\
            \end{center}
            
            \begin{itemize}
                \item $(1-n_1) = 1$ when the farmer is on the goal shore and thus has to go back and get another object from the starting shore. $(1-n_1) = 0$ if he is already on the starting shore and thus doesn't have to travel back to the starting shore to get another object.
                \item $2(n_2 + n_3 + n_4) - 1$: for each object he brings from the starting shore he has to travel back except for the last one he brings to the other side.
            \end{itemize}
        
            \item Our heuristic is defined to calculate the optimal number of actions in the simplified search problem. Thus obviously $h(n) = h^*(n)$, and therefore admissible in the simplified search problem. Thus in the original search problem $h(n) \leq h^*(n)$ and therefore admissible.
        
            \begin{center}
                $h(n) \leq c(n,a,n') + h(n')$
            \end{center}
        
            \item We have to show that the triangle equation applies. $c(n,a,n') = 1$ as we are counting actions, and the cost of doing one action is $1$. To accept the triangle equation $h(n')$ can not change by more that $1$ so we have to show $||h(n') - h(n)|| \leq 1$. We can reduce this to two cases:
            \begin{itemize}
                \item If just the farmer changes side, everything except $n_1$ is constant. Additionally, $||{n'}_1 - n_1 || \leq 1$ by definition. \\
                $||h(n') - h(n)|| = ||(1 - {n'}_1) + 2(n_2 + n_3 + n_4) - 1) - ((1 - n_1) + 2(n_2 + n_3 + n_4) - 1)|| = ||n_1 - {n'}_1|| \leq 1$ \\
                \item If the farmer and a object $i$ changes site, 
                everything except $n_1$ and $n_i $ are constant also $n_1 = n_i$ as both change to the same site. Additionally, $||{n'}_1 - n_1 || \leq 1$ by definition. We choose, $x_i = 2$, as it is equivalent for any of the other $x_i$ with $1 < i \leq 4$ as they are symmetric. \\
                $||h(n') - h(n)|| = ||(1 - {n'}_1) + 2({n'}_1 + n_3 + n_4) - 1) - ((1 - n_1) + 2(n_1 + n_3 + n_4) - 1)|| = ||n_1 - {n'}_1 + 2{n'}_1 - 2n_1 || = || n_1 - {n'}_1 || = 1$ \\
            \end{itemize}
        
            $\implies ||h(n') - h(n)|| \leq 1 \implies h(n)$ is consistent.
        
            
        
        
        \end{subexercises}
    \end{exercise}

\begin{exercise}[Informed Search Algorithms: Practice]{2}
\begin{subexercises}
\item A* without reopening  \\\\
The numbers on the nodes denote the order in which they would be expanded.\\
\begin{tikzpicture}[roundnode/.style={align=center,ellipse,draw=gray!60,very thick, minimum size=8mm,node distance=17pt, scale=0.8}] 

%NODES

\node[roundnode] (basel)  {1 \\ Basel \\ f=38 \\ g=0 \\ h=83};
\node[roundnode] (olten)[below=of basel] {2 \\ Olten\\ f=99 \\ g=48 \\ h=51};
\node[roundnode] (baden)  [below=of basel, right=of olten] {4 \\ Baden\\ f=108 \\\ g=70 \\ h=38};

\node[roundnode] (aarau) [below=of olten] {3 \\ Aarau \\ f=106 \\ g=62 \\ h=44};
\node[roundnode] (basel1) [below=of olten, left=of aarau] {\\ Basel \\ f=179 \\ g=96 \\ h=83};
\node[roundnode] (luzern) [below=of olten, left=of basel1] {6 \\ Luzern \\ f=124  \\ g=103 \\ h=21};

\node[roundnode] (luzern1) [below=of aarau] {\\ Luzern \\ f=133 \\ g=112 \\ h=21};
\node[roundnode] (olten1) [below=of aarau, left=of luzern1] {\\ Olten \\ f=127 \\ g=76 \\ h=51};
\node[roundnode] (baden1) [below=of aarau, right=of luzern1] {\\ Baden \\ f=127 \\ g=89 \\ h=38};


%Expand from Baden
\node[roundnode] (aarau1) [below=of baden] {\\ Aarau \\ f=141 \\ g=97 \\ h=44};
\node[roundnode] (zurich) [below=of baden, right=of aarau1] {5 \\ Zürich \\ f=117 \\ g=94 \\ h=23};
\node[roundnode] (basel2) [below=of baden, right=of zurich] {\\ Basel \\ f=223 \\ g=140 \\ h=83};


%Expand from Zürich
\node[roundnode] (baden2) [below=of zurich] {\\ Baden \\ f=156 \\ g=118 \\ h=38};
\node[roundnode] (zug1) [below=of zurich, right=of baden2] {7 \\ Zug \\ f=126 \\ g=126 \\ h=0};


%Expand from Lucerne
\node[roundnode] (olten2) [below=of luzern] {\\ Olten \\ f=209 \\ g=158 \\ h=51};
\node[roundnode] (aarau2) [below=of luzern, left=of olten2] {\\ Aarau \\ f=197 \\ g=153 \\ h=44};
\node[roundnode] (zug) [below=of luzern, left=of aarau2] {\\ Zug \\ f=133 \\  g=133 \\ h=0};


%LINES
\draw[-](basel.south) --(olten.north);
\draw[-](basel.south) --(baden.north);
\draw[-](olten.south) --(aarau.north);
\draw[-](olten.south) --(luzern.north);
\draw[-](olten.south) --(basel1.north);
\draw[-](aarau.south) --(olten1.north);
\draw[-](aarau.south) --(baden1.north);
\draw[-](aarau.south) --(luzern1.north);
\draw[-](luzern.south) --(olten2.north);
\draw[-](luzern.south) --(aarau2.north);
\draw[-](luzern.south) --(zug.north);
\draw[-](baden.south) --(aarau1.north);
\draw[-](baden.south) --(zurich.north);
\draw[-](baden.south) --(basel2.north);
\draw[-](zurich.south)--(baden2.north);
\draw[-](zurich.south)--(zug1.north);
\
\end{tikzpicture}
\item Greedy Best First Search Algorithm \\

\begin{tikzpicture}[roundnode/.style={align=center,ellipse,draw=gray!60,very thick, minimum size=8mm,node distance=15pt}] 
\node[roundnode] (basel)  {1 \\ Basel \\ h=83};
\node[roundnode] (olten)[below=of basel] {  \\ Olten \\ h=51};
\node[roundnode] (baden)  [below=of basel, right=of olten] {2 \\ Baden\\ h=38};
\draw[-](basel.south) --(olten.north);
\draw[-](basel.south) --(baden.north);

%Expand from Baden
\node[roundnode] (aarau1) [below=of baden] {\\ Aarau \\ h=44};
\node[roundnode] (zurich) [below=of baden, right=of aarau1] {3 \\ Zürich \\ h=23};
\node[roundnode] (basel2) [below=of baden, right=of zurich] {\\ Basel \\ h=83};
\draw[-](baden.south) --(aarau1.north);
\draw[-](baden.south) --(zurich.north);
\draw[-](baden.south) --(basel2.north);

%Expand from Zürich
\node[roundnode] (baden2) [below=of zurich] {\\ Baden \\ h=38};
\node[roundnode] (zug1) [below=of zurich, right=of baden2] {4 \\ Zug \\ h=0};
\draw[-](zurich.south)--(baden2.north);
\draw[-](zurich.south)--(zug1.north);

\end{tikzpicture}
\item Comparison \\
\\In this case it is visible that the heuristics were chosen well, such that the greedy algorithm was faster while generating less nodes and still arriving to the optimal solution. The Greedy-Best-First-Search Algorithm only guarantees to arrive to a solution, but it is not guaranteed to find the optimal solution, but rather the first one it finds.
\end{subexercises}   
\end{exercise}
\begin{exercise}[Informed Search Programming]{3}
    \begin{subexercises}
 
        \item [b)] Analysis \\ 
       
    \end{subexercises}
\end{exercise}



\end{document}
