\documentclass[11pt]{scrartcl}

\usepackage[utf8]{inputenc}
\usepackage[T1]{fontenc}
\usepackage[english]{babel}
\usepackage{lmodern}
\usepackage{graphicx}
\usepackage{listings}
\usepackage{xspace}
\usepackage[a4paper,lmargin={2cm},rmargin={2cm},tmargin={2.5cm},bmargin = {2.5cm},headheight = {4cm}]{geometry}
\usepackage{amsmath,amssymb,amstext,amsthm}
\usepackage[shortlabels]{enumitem}
\usepackage[headsepline]{scrlayer-scrpage} 
\pagestyle{scrheadings} 
\usepackage{titling}
\usepackage{etoolbox}
\usepackage{tikz}
\usetikzlibrary{shapes, arrows, calc, automata, arrows.meta, positioning,decorations.pathmorphing,backgrounds,decorations.markings,decorations.pathreplacing, graphs}

\tikzset{% 
    initial text={},
    state/.style={circle, draw, minimum size=.6cm},
    every initial by arrow/.style={-stealth},
    every loop/.append style={-stealth},
    >=stealth
}

\ohead{\theauthor}
\ihead{Introduction to Artificial Intelligence, Spring 2022, Sheet \thesheetnr}

% Sheet number
\newcounter{sheetnr}
\newcommand{\sheetnr}[1]{\setcounter{sheetnr}{#1}}

% Exercise environments
\newenvironment{exercise}[2][]{\section*{Exercise \thesheetnr.#2\expandafter\ifstrempty\expandafter{#1}{}{\ (#1)}}}{}
\newenvironment{subexercises}{\begin{enumerate}[a), font=\bfseries, wide, labelindent=0pt]}{\end{enumerate}}
\newenvironment{subsubexercises}{\begin{enumerate}[i), font=\bfseries, wide, labelindent=0pt]}{\end{enumerate}}



%% Examples
\newcommand{\Reach}{\problem{Reach}}


% Anpassen --> %
\author{Oliver Strassmann \\
        Julia Kostadinova \\
        Alessio Brazerol}
\sheetnr{3}
% <-- Anpassen %

\begin{document}

\begin{exercise}[Heuristics]{1}
    \begin{subexercises}
        \item Defintion of Heuristic  \\ 
        \item Admissible Heuristic  \\
        \item Consitent Heuristic \\ 
       
    \end{subexercises}
\end{exercise}

\begin{exercise}[Informed Search Algorithms: Practice]{2}
\begin{subexercises}
\item A* without reopening  \\\\
The numbers on the nodes denote the order in which they would be expanded.\\
\begin{tikzpicture}[roundnode/.style={align=center,ellipse,draw=gray!60,very thick, minimum size=8mm,node distance=17pt, scale=0.8}] 

%NODES

\node[roundnode] (basel)  {1 \\ Basel \\ f=38 \\ g=0 \\ h=83};
\node[roundnode] (olten)[below=of basel] {2 \\ Olten\\ f=99 \\ g=48 \\ h=51};
\node[roundnode] (baden)  [below=of basel, right=of olten] {4 \\ Baden\\ f=108 \\\ g=70 \\ h=38};

\node[roundnode] (aarau) [below=of olten] {3 \\ Aarau \\ f=106 \\ g=62 \\ h=44};
\node[roundnode] (basel1) [below=of olten, left=of aarau] {\\ Basel \\ f=179 \\ g=96 \\ h=83};
\node[roundnode] (luzern) [below=of olten, left=of basel1] {6 \\ Luzern \\ f=124  \\ g=103 \\ h=21};

\node[roundnode] (luzern1) [below=of aarau] {\\ Luzern \\ f=133 \\ g=112 \\ h=21};
\node[roundnode] (olten1) [below=of aarau, left=of luzern1] {\\ Olten \\ f=127 \\ g=76 \\ h=51};
\node[roundnode] (baden1) [below=of aarau, right=of luzern1] {\\ Baden \\ f=127 \\ g=89 \\ h=38};


%Expand from Baden
\node[roundnode] (aarau1) [below=of baden] {\\ Aarau \\ f=141 \\ g=97 \\ h=44};
\node[roundnode] (zurich) [below=of baden, right=of aarau1] {5 \\ Zürich \\ f=117 \\ g=94 \\ h=23};
\node[roundnode] (basel2) [below=of baden, right=of zurich] {\\ Basel \\ f=223 \\ g=140 \\ h=83};


%Expand from Zürich
\node[roundnode] (baden2) [below=of zurich] {\\ Baden \\ f=156 \\ g=118 \\ h=38};
\node[roundnode] (zug1) [below=of zurich, right=of baden2] {7 \\ Zug \\ f=126 \\ g=126 \\ h=0};


%Expand from Lucerne
\node[roundnode] (olten2) [below=of luzern] {\\ Olten \\ f=209 \\ g=158 \\ h=51};
\node[roundnode] (aarau2) [below=of luzern, left=of olten2] {\\ Aarau \\ f=197 \\ g=153 \\ h=44};
\node[roundnode] (zug) [below=of luzern, left=of aarau2] {\\ Zug \\ f=133 \\  g=133 \\ h=0};


%LINES
\draw[-](basel.south) --(olten.north);
\draw[-](basel.south) --(baden.north);
\draw[-](olten.south) --(aarau.north);
\draw[-](olten.south) --(luzern.north);
\draw[-](olten.south) --(basel1.north);
\draw[-](aarau.south) --(olten1.north);
\draw[-](aarau.south) --(baden1.north);
\draw[-](aarau.south) --(luzern1.north);
\draw[-](luzern.south) --(olten2.north);
\draw[-](luzern.south) --(aarau2.north);
\draw[-](luzern.south) --(zug.north);
\draw[-](baden.south) --(aarau1.north);
\draw[-](baden.south) --(zurich.north);
\draw[-](baden.south) --(basel2.north);
\draw[-](zurich.south)--(baden2.north);
\draw[-](zurich.south)--(zug1.north);
\
\end{tikzpicture}
\item Greedy Best First Search Algorithm \\

\begin{tikzpicture}[roundnode/.style={align=center,ellipse,draw=gray!60,very thick, minimum size=8mm,node distance=15pt}] 
\node[roundnode] (basel)  {1 \\ Basel \\ h=83};
\node[roundnode] (olten)[below=of basel] {  \\ Olten \\ h=51};
\node[roundnode] (baden)  [below=of basel, right=of olten] {2 \\ Baden\\ h=38};
\draw[-](basel.south) --(olten.north);
\draw[-](basel.south) --(baden.north);

%Expand from Baden
\node[roundnode] (aarau1) [below=of baden] {\\ Aarau \\ h=44};
\node[roundnode] (zurich) [below=of baden, right=of aarau1] {3 \\ Zürich \\ h=23};
\node[roundnode] (basel2) [below=of baden, right=of zurich] {\\ Basel \\ h=83};
\draw[-](baden.south) --(aarau1.north);
\draw[-](baden.south) --(zurich.north);
\draw[-](baden.south) --(basel2.north);

%Expand from Zürich
\node[roundnode] (baden2) [below=of zurich] {\\ Baden \\ h=38};
\node[roundnode] (zug1) [below=of zurich, right=of baden2] {4 \\ Zug \\ h=0};
\draw[-](zurich.south)--(baden2.north);
\draw[-](zurich.south)--(zug1.north);

\end{tikzpicture}
\item Comparison \\
\\In this case it is visible that the heuristics were chosen well, such that the greedy algorithm was faster while generating less nodes and still arriving to the optimal solution. The Greedy-Best-First-Search Algorithm only guarantees to arrive to a solution, but it is not guaranteed to find the optimal solution, but rather the first one it finds.
\end{subexercises}   
\end{exercise}
\begin{exercise}[Informed Search Programming]{3}
    \begin{subexercises}
 
        \item [b)] Analysis \\ 
       
    \end{subexercises}
\end{exercise}



\end{document}