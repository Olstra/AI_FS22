\documentclass[11pt]{scrartcl}

\usepackage[utf8]{inputenc}
\usepackage[T1]{fontenc}
\usepackage[english]{babel}
\usepackage{lmodern}
\usepackage{graphicx}
\usepackage{listings}
\usepackage{xspace}
\usepackage[a4paper,lmargin={2cm},rmargin={2cm},tmargin={2.5cm},bmargin = {2.5cm},headheight = {4cm}]{geometry}
\usepackage{amsmath,amssymb,amstext,amsthm}
\usepackage[shortlabels]{enumitem}
\usepackage[headsepline]{scrlayer-scrpage} 
\pagestyle{scrheadings} 
\usepackage{titling}
\usepackage{etoolbox}
\usepackage{tikz}
\usetikzlibrary{shapes, arrows, calc, automata, arrows.meta, positioning,decorations.pathmorphing,backgrounds,decorations.markings,decorations.pathreplacing, graphs}

\tikzset{% 
    initial text={},
    state/.style={circle, draw, minimum size=.6cm},
    every initial by arrow/.style={-stealth},
    every loop/.append style={-stealth},
    >=stealth
}

\ohead{\theauthor}
\ihead{Introduction to Artificial Intelligence, Spring 2022, Sheet \thesheetnr}

% Sheet number
\newcounter{sheetnr}
\newcommand{\sheetnr}[1]{\setcounter{sheetnr}{#1}}

% Exercise environments
\newenvironment{exercise}[2][]{\section*{Exercise \thesheetnr.#2\expandafter\ifstrempty\expandafter{#1}{}{\ (#1)}}}{}
\newenvironment{subexercises}{\begin{enumerate}[a), font=\bfseries, wide, labelindent=0pt]}{\end{enumerate}}
\newenvironment{subsubexercises}{\begin{enumerate}[i), font=\bfseries, wide, labelindent=0pt]}{\end{enumerate}}



%% Examples
\newcommand{\Reach}{\problem{Reach}}


% Anpassen --> %
\author{Oliver Strassmann \\
        Julia Kostadinova \\
        Alessio Brazerol}
\sheetnr{3}
% <-- Anpassen %

\begin{document}

\begin{exercise}[Heuristics]{1}
    \begin{subexercises}
        \item Defintion of Heuristic  \\ 
        \item Admissible Heuristic  \\
        \item Consitent Heuristic \\ 
       
    \end{subexercises}
\end{exercise}

\begin{exercise}[Informed Search Algorithms: Practice]{2}
\begin{subexercises}
\item A* without reopening  \\

\begin{tikzpicture}[roundnode/.style={align=center,ellipse,draw=gray!60,very thick, minimum size=10mm}] 
\centering
%NODES

\node[roundnode] (basel) {Basel \\ 38=0+38};
\node[roundnode] (olten)[below=of basel] {Olten\\ 99=48+51};
\node[roundnode] (baden)  [below=of basel,right=of olten] {Baden\\ 108=70+38};

\node[roundnode] (aarau) [below=of olten] {Aarau \\ 106=62+44};
\node[roundnode] (basel1) [below=of olten, right=of aarau] {Basel \\ 179=96+83};
\node[roundnode] (luzern) [below=of olten, right=of basel1] {Luzern \\ 124=103+21};

\node[roundnode] (olten1) [below=of aarau] {Olten \\ 127=76+51};
\node[roundnode] (baden1) [below=of aarau, right=of olten1] {Baden \\ 127=89+38};
\node[roundnode] (luzern1) [below=of aarau, right=of baden1] {Luzern \\ 133=112+21};

\node[roundnode] (olten2) [below=of luzern, right=of luzern1] {Olten \\ 209=158+51};
\node[roundnode] (aarau2) [below=of luzern, right=of olten2] {Aarau \\ 197=153+44};
\node[roundnode] (zug) [below=of luzern, right=of aarau2] {Zug \\ 142=142+0};


%LINES
\draw[-](basel.south) --(olten.north);
\draw[-](basel.south) --(baden.north);
\draw[-](olten.south) --(aarau.north);
\draw[-](olten.south) --(luzern.north);
\draw[-](olten.south) --(basel1.north);
\draw[-](aarau.south) --(olten1.north);
\draw[-](aarau.south) --(baden1.north);
\draw[-](aarau.south) --(luzern1.north);
\draw[-](luzern.south) --(olten2.north);
\draw[-](luzern.south) --(aarau2.north);

\end{tikzpicture}
\end{subexercises}   
\end{exercise}




\end{document}